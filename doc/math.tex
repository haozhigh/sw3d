\documentclass[12pt]{article}

\usepackage[charter]{mathdesign}
\usepackage[bookmarks,bookmarksopen]{hyperref}
\usepackage{bm}

\begin{document}
\section{Fixed Functions in Graphic Pipeline}
    \subsection{Rasterization}
        Addition Fomulars:
        \[ \begin{array}{c}
            \cos{ \left( x+y \right) }=\cos{x}\cos{y}-\sin{x}\sin{y} \\
            \sin{ \left( x+y \right) }=\sin{x}\cos{y}+\cos{x}\sin{y}
        \end{array} \]
        2D rotate matrix:
        \[ \left[ \begin{array}{rr}
            \cos{\alpha} & -\sin{\alpha} \\
            \sin{\alpha} & \cos{\alpha}
        \end{array} \right] \]
        2D matrix to rotate a vector 90 degress counter-clockwise:
        \[ \left[ \begin{array}{rr}
            0 & 0\\
            -1 & 0
        \end{array} \right] \]
        To judge vector v1 is pointing to the right side of vector v0:
        \[
        \left(
            \left[ \begin{array}{rr} 0 & 0\\ -1 & 0 \end{array} \right]
            \overrightarrow{v_0}
        \right)
        \cdot \overrightarrow{v_1} > 0
        \]
        In OpenGL, default visiable triangles are counter-clockwise, thus left side of 
        three edges form the triangle aera.
    \subsection{Coordinate System Transform}
        Describe (u,v,w) space axis in (x,y,z) space:
        \begin{eqnarray*}
            \left[ \begin{array}{ccc} \overrightarrow{u} & \overrightarrow{v} & \overrightarrow{w} \end{array} \right] &
            = &
            \left[ \begin{array}{ccc} \overrightarrow{x} & \overrightarrow{y} & \overrightarrow{z} \end{array} \right]
            \times
            \left[ \begin{array}{ccc}
                x_u & x_v & x_w \\
                y_u & y_v & y_w \\
                z_u & z_v & z_w
            \end{array} \right] \\
            & = &
            \left[ \begin{array}{ccc} \overrightarrow{x} & \overrightarrow{y} & \overrightarrow{z} \end{array} \right]
            \times
            \textbf{P}
        \end{eqnarray*}
        \textbf{P} converts a vector from (u,v,w) space to (x,y,z) space; Inv(\textbf{P})
        converts a vector from (x,y,z) space to (u,v,w) space:
        \begin{eqnarray*}
            \left[ \begin{array}{c} u_0 \\ v_0 \\ w_0 \end{array} \right] &
            = &
            \left[ \begin{array}{ccc}
                \overrightarrow{u} & \overrightarrow{v} & \overrightarrow{w}
            \end{array} \right]
            \times
            \left[ \begin{array}{c} u_0 \\ v_0 \\ w_0 \end{array} \right] \\
            & = &
            \left[ \begin{array}{ccc}
            \overrightarrow{x} & \overrightarrow{y} & \overrightarrow{z}
            \end{array} \right]
            \times
            \textbf{P}
            \times
            \left[ \begin{array}{c} u_0 \\ v_0 \\ w_0 \end{array} \right] \\
            & = &
            \left[ \begin{array}{ccc}
            \overrightarrow{x} & \overrightarrow{y} & \overrightarrow{z}
            \end{array} \right]
            \times
            \left(
                \textbf{P}
                \times
                \left[ \begin{array}{c} u_0 \\ v_0 \\ w_0 \end{array} \right]
            \right)
        \end{eqnarray*}
        For points transformation:
        \[
            \left[ \begin{array}{cc}
                \textbf{P} & \begin{array}{c} u_{root} \\ v_{root} \\ w_{root} \end{array} \\
                \begin{array}{ccc} 0 & 0 & 0 \end{array} & 1
            \end{array} \right]
            \times
            \left[ \begin{array}{cc}
                \bm{P^{-1}} &
                -\bm{P^{-1}} \times
                \left[ \begin{array}{c}
                    u_{root} \\ v_{root} \\ w_{root}
                \end{array} \right] \\
                \begin{array}{ccc} 0 & 0 & 0 \end{array} &
                1
            \end{array} \right]
            =
            \left[ \begin{array}{cccc}
                1 & 0 & 0 & 0 \\
                0 & 1 & 0 & 0 \\
                0 & 0 & 1 & 0 \\
                0 & 0 & 0 & 1
            \end{array} \right]
        \]
\section{Linear Algebra}
    \subsection{Cross Product}
    Cross product definition differs in right-hand coordinate system and left-hand
    coordinate system ensuring that:
    \[ \begin{array}{c}
        \overrightarrow{x} \times \overrightarrow{y} = \overrightarrow{z} \\
        \overrightarrow{y} \times \overrightarrow{z} = \overrightarrow{x} \\
        \overrightarrow{z} \times \overrightarrow{x} = \overrightarrow{y}
    \end{array} \]
    From this, it can be inferred that:
    \begin{eqnarray*}
        \left[ \begin{array}{c}
            a_1 \\ a_2 \\ a_3
        \end{array} \right]
        \times
        \left[ \begin{array}{c}
            b_1 \\ b_2 \\ b_3
        \end{array} \right] &
        = &
        \left(
            a_1\overrightarrow{x} + a_2\overrightarrow{y} + a_3\overrightarrow{z}
        \right)
        \times
        \left(
            a_1\overrightarrow{x} + a_2\overrightarrow{y} + a_3\overrightarrow{z}
        \right) \\
        & = &
        \left| \begin{array}{ccc}
            \overrightarrow{x} & \overrightarrow{y} & \overrightarrow{z} \\
            a_1 & a_2 & a_3 \\
            b_1 & b_2 & b_3
        \end{array} \right|
    \end{eqnarray*}
\end{document}